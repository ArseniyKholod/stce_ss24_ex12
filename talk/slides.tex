\documentclass[ucs,10pt]{beamer}
 
% Template for talks using the Logo of STCE and Corporate Design of RWTH Aachen
% adapted from:
% https://www.mi.fu-berlin.de/w/Mi/BeamerTemplateCorporateDesign

\usepackage{amsmath,dsfont,listings}
% ** for picture **
\usepackage{graphicx}

%%% STCE logo
% small version for upper right corner of normal pages
\pgfdeclareimage[height=0.9cm]{university-logo}{rwth_i12_softw-werkz_en_rgb.png}
\logo{\pgfuseimage{university-logo}}
% large version for upper right corner of title page
\pgfdeclareimage[height=1cm]{big-university-logo}{rwth_i12_softw-werkz_en_rgb.png}
\newcommand{\titleimage}[1]{\pgfdeclareimage[height=2cm]{title-image}{#1}}
\titlegraphic{\pgfuseimage{title-image}}
%%% end STCE logo

% NOTE: 1cm = 0.393 in = 28.346 pt;    1 pt = 1/72 in = 0.0352 cm
\setbeamersize{text margin right=3.5mm, text margin left=7.5mm}  % text margin

% colors to be used
\definecolor{text-grey}{RGB}{51, 51, 51} % grey text on white background
\definecolor{bg-grey}{rgb}{0.66, 0.65, 0.60} % grey background (for white text)
\definecolor{rwth-blue}{RGB}{0, 83, 159} % blue text
\definecolor{rwth-green}{RGB}{153, 204, 0} % green text
\definecolor{rwth-red}{RGB}{204, 0, 0} % red text (used by \alert)

% switch off the sidebars
% TODO: loading \useoutertheme{sidebar} (which is maybe wanted) also inserts
%   a sidebar on title page (unwanted), also indents the page title (unwanted?),
%   and duplicates the navigation symbols (unwanted)
\setbeamersize{sidebar width left=0cm, sidebar width right=0mm}
\setbeamertemplate{sidebar right}{}
\setbeamertemplate{sidebar left}{}
%    XOR
% \useoutertheme{sidebar}

% frame title
% is truncated before logo and splits on two lines
% if neccessary (or manually using \\)
\setbeamertemplate{frametitle}{%
    \vskip-30pt \color{purple}\large%
    \begin{minipage}[b][23pt]{80.5mm}%
    \flushleft\insertframetitle%
    \end{minipage}%
}

%%% title page
% TODO: get rid of the navigation symbols on the title page.
%   actually, \frame[plain] *should* remove them...
\setbeamertemplate{title page}{
% upper right: STCE logo
\vskip2pt\hfill\pgfuseimage{big-university-logo} \\
\vskip6pt\hskip3pt
% title image of the presentation
% set the title and the author
\begin{center}
\vskip4pt
\large \inserttitle \vskip5pt  \small \insertsubtitle
\vskip8pt
	\normalsize \insertauthor %\\ 
	%\includegraphics[width=2cm]{../../foto_naumann}	
\\ [5mm]
	\footnotesize \insertinstitute 
\end{center}
}
%%% end title page

%%% colors
\usecolortheme{lily}
\setbeamercolor*{normal text}{fg=black,bg=white}
\setbeamercolor*{alerted text}{fg=rwth-red}
\setbeamercolor*{example text}{fg=rwth-green}
\setbeamercolor*{structure}{fg=rwth-blue}

\setbeamercolor*{block title}{fg=white,bg=black!50}
\setbeamercolor*{block title alerted}{fg=white,bg=black!50}
\setbeamercolor*{block title example}{fg=white,bg=black!50}

\setbeamercolor*{block body}{bg=black!10}
\setbeamercolor*{block body alerted}{bg=black!10}
\setbeamercolor*{block body example}{bg=black!10}

\setbeamercolor{bibliography entry author}{fg=rwth-blue}
% TODO: this doesn't work at all:
\setbeamercolor{bibliography entry journal}{fg=text-grey}

\setbeamercolor{item}{fg=rwth-blue}
\setbeamercolor{navigation symbols}{fg=text-grey,bg=bg-grey}
%%% end colors

%%% headline
\setbeamertemplate{headline}{
\vskip4pt\hfill\insertlogo\hspace{3.5mm} % logo on the right

\vskip6pt\color{rwth-blue}\rule{\textwidth}{0.4pt} % horizontal line
}
%%% end headline

%%% footline
\newcommand{\footlinetext}{\insertshortinstitute, \insertshorttitle}
\setbeamertemplate{footline}{
\vskip5pt\color{rwth-blue}\rule{\textwidth}{0.4pt}\\ % horizontal line
\vskip2pt
\makebox[123mm]{\hspace{7.5mm}
\color{rwth-blue}\footlinetext
\hfill \raisebox{-1pt}{\usebeamertemplate***{navigation symbols}}
\hfill \insertframenumber}
\vskip4pt
}
%%% end footline

%%% settings for listings package
\lstset{extendedchars=true, showstringspaces=false, basicstyle=\footnotesize\sffamily, tabsize=2, breaklines=true, breakindent=10pt, frame=l, columns=fullflexible}
\lstset{language=C++} % this sets the syntax highlighting
\lstset{mathescape=true} % this switches on $...$ substitution in code
% enables UTF-8 in source code:
\lstset{literate={ä}{{\"a}}1 {ö}{{\"o}}1 {ü}{{\"u}}1 {Ä}{{\"A}}1 {Ö}{{\"O}}1 {Ü}{{\"U}}1 {ß}{\ss}1}
%%% end listings
  

\begin{document}
\title[{\tt info@stce.rwth-aachen.de}]{\textcolor{rwth-blue}{Software Lab Computational Engineering Science} \vspace{.2cm} \\ {\small Group 12, Exception Handling}}
\author[Group 12)]{Aaron Floerke, Arseniy Kholod, Xinyang Song and Yanliang Zhu} 
\institute[Software Lab CES]{
{Informatik 12: Software and Tools for Computational Engineering (STCE)} \\ RWTH Aachen University \vspace{.5cm}
}
\date[]{24.06.2024}

\begin{frame}[plain]
\titlepage
\end{frame}

\begin{frame}
	\frametitle{Contents}
\tableofcontents
\end{frame}

\section{Preface}

\begin{frame}
\frametitle{Preface \\
	\small \color{rwth-blue} Exception Handling}

	\begin{itemize}
		\item Software always has a working domain.
		\item User of the software is not aware of all limitations.
		\item Software developer helps user by introducing appropriate exception handling.
		\item Our task is to introduce an exception handling to cppNum v2.4 and v2.5.
	\end{itemize}
\end{frame}

\section{Analysis}

\subsection{User Requirements}

\begin{frame}
\frametitle{Analysis \\
	\small \color{rwth-blue} User Requirements}
	\begin{itemize}
		\item Extend cppNum v2.4 and v2.5 with appropriate C++ exception handling.
		\item Desing at least three scalable sufficiently distinct case studies.
		\item Compare general behavior and run times with the exception handling-free version.
	\end{itemize}
\end{frame}

\subsection{System Requirements}

\begin{frame}
\frametitle{Analysis \\
	\small \color{rwth-blue} System Requirements}
	Functional:
	\begin{itemize}
		\item An exception is thrown, if system is not able to produce correct result, because input data are incorrect or outside the domain.
		\item An exception is thrown before potential crash of the system.
		\item An exception contains an explanatory string.
	\end{itemize}
	Nonfunctional:
	\begin{itemize}
		\item An exception is a class object.
		\item All cppNum exception classes have a single parent class to provide clear structure.
		\item All exception classes are inherited from std::exception to catch together with other exceptions, potentially generated by thirdparty libraries.
	\end{itemize}
\end{frame}

\section{Design}

\subsection{System Requirements}

\begin{frame}
\frametitle{Design \\
	\small \color{rwth-blue} Principal Components and Third-Party Software}
	\begin{itemize}
		\item std::exception to enherit exception classes from.
		\item Eigen library to prove applicability of LU and LLT decompositions.
		\item throw and try-catch mechanism to throw and catch exceptions.
	\end{itemize}
\end{frame}

\subsection{Class Model(s)}

\begin{frame}
\frametitle{Design \\
	\small \color{rwth-blue} Class Model(s)}
\end{frame}

\section{Implementation}

\subsection{Development Infrastructure}

\begin{frame}
\frametitle{Implementation \\
	\small \color{rwth-blue} Development Infrastructure}
\end{frame}

\subsection{Source Code}

\begin{frame}
\frametitle{Implementation \\
	\small \color{rwth-blue} Source Code}
\end{frame}

\subsection{Software Tests}

\begin{frame}
\frametitle{Implementation \\
	\small \color{rwth-blue} Software Tests}
\end{frame}

\section{Project Management}

\begin{frame}
\frametitle{Project Management}
\end{frame}

\section{Live Software Demo}

\begin{frame}
\frametitle{Live Software Demo} 
\end{frame}

\section{Summary and Conclusion}

\begin{frame}
\frametitle{Summary and Conclusion}
\end{frame}

\end{document}
